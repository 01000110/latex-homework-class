\documentclass[11pt,largemargins]{homework}

\newcommand{\hwname}{آلن تورینگ}
\newcommand{\hwsn}{۹۱۳۱۰۰۰}
\newcommand{\hwtype}{تمرین}
\newcommand{\hwnum}{اول}
\newcommand{\hwclass}{ریاضیات گسسته}

\begin{document}
\maketitle

\question
  این جواب سوال اول است.

  اطلاعات شخصی و کلاس خود را در بالا وارد کنید!

\question
  شماره سوالات به طور خودکار افزایش مییابد.

  لورم ایپسوم یا طرح‌نما (به انگلیسی: Lorem ipsum) به متنی آزمایشی و بی‌معنی در صنعت چاپ، صفحه‌آرایی و طراحی گرافیک گفته می‌شود. طراح گرافیک از این متن به عنوان عنصری از ترکیب بندی برای پر کردن صفحه و ارایه اولیه شکل ظاهری و کلی طرح سفارش گرفته شده استفاده می نماید، تا از نظر گرافیکی نشانگر چگونگی نوع و اندازه فونت و ظاهر متن باشد. معمولا طراحان گرافیک برای صفحه‌آرایی، نخست از متن‌های آزمایشی و بی‌معنی استفاده می‌کنند تا صرفا به مشتری یا صاحب کار خود نشان دهند که صفحه طراحی یا صفحه بندی شده بعد از اینکه متن در آن قرار گیرد چگونه به نظر می‌رسد و قلم‌ها و اندازه‌بندی‌ها چگونه در نظر گرفته شده‌است. از آنجایی که طراحان عموما نویسنده متن نیستند و وظیفه رعایت حق تکثیر متون را ندارند و در همان حال کار آنها به نوعی وابسته به متن می‌باشد آنها با استفاده از محتویات ساختگی، صفحه گرافیکی خود را صفحه‌آرایی می‌کنند تا مرحله طراحی و صفحه‌بندی را به پایان برند.

\question*{سوالی با عنوان عجیب و غریب}
  این سوال عنوان عجیبی دارد!

لورم ایپسوم یا طرح‌نما (به انگلیسی: Lorem ipsum) به متنی آزمایشی و بی‌معنی در صنعت چاپ، صفحه‌آرایی و طراحی گرافیک گفته می‌شود. طراح گرافیک از این متن به عنوان عنصری از ترکیب بندی برای پر کردن صفحه و ارایه اولیه شکل ظاهری و کلی طرح سفارش گرفته شده استفاده می نماید، تا از نظر گرافیکی نشانگر چگونگی نوع و اندازه فونت و ظاهر متن باشد. معمولا طراحان گرافیک برای صفحه‌آرایی، نخست از متن‌های آزمایشی و بی‌معنی استفاده می‌کنند تا صرفا به مشتری یا صاحب کار خود نشان دهند که صفحه طراحی یا صفحه بندی شده بعد از اینکه متن در آن قرار گیرد چگونه به نظر می‌رسد و قلم‌ها و اندازه‌بندی‌ها چگونه در نظر گرفته شده‌است. از آنجایی که طراحان عموما نویسنده متن نیستند و وظیفه رعایت حق تکثیر متون را ندارند و در همان حال کار آنها به نوعی وابسته به متن می‌باشد آنها با استفاده از محتویات ساختگی، صفحه گرافیکی خود را صفحه‌آرایی می‌کنند تا مرحله طراحی و صفحه‌بندی را به پایان برند.

\question
با استفاده از محیط  
   \texttt{induction} 
   به سادگی میتوان اثباتهای استقرایی را اضافه کرد!
  \begin{induction}
    \basecase
      درستی حکم را برای $n=1$ بررسی میکنیم. 
    \indhyp
     فرض استقرا را اینجا اضافه میکنیم. مثلاً یک فرمول ریاضی هم دارد:
      \[a^2 + b^2 = c^2\]
    \indstep
نشان میدهیم اگر فرض استقرا برای $k$ های کوچکتر از $n$ برقرار باشد برای $n+1$ نیز برقرار است.
  \end{induction}

  و لذا حکم برقرار است. 

\question
از محیط   
  \texttt{arabicparts}
برای اضافه شدن شماره سوال در لیست استفاده کنید.  
  \begin{arabicparts}
    \item استفاده از \LaTeX
    \item ???
    \item مفید است!
  \end{arabicparts}

  لورم ایپسوم یا طرح‌نما (به انگلیسی: Lorem ipsum) به متنی آزمایشی و بی‌معنی در صنعت چاپ، صفحه‌آرایی و طراحی گرافیک گفته می‌شود. طراح گرافیک از این متن به عنوان عنصری از ترکیب بندی برای پر کردن صفحه و ارایه اولیه شکل ظاهری و کلی طرح سفارش گرفته شده استفاده می نماید، تا از نظر گرافیکی نشانگر چگونگی نوع و اندازه فونت و ظاهر متن باشد. معمولا طراحان گرافیک برای صفحه‌آرایی، نخست از متن‌های آزمایشی و بی‌معنی استفاده می‌کنند تا صرفا به مشتری یا صاحب کار خود نشان دهند که صفحه طراحی یا صفحه بندی شده بعد از اینکه متن در آن قرار گیرد چگونه به نظر می‌رسد و قلم‌ها و اندازه‌بندی‌ها چگونه در نظر گرفته شده‌است. از آنجایی که طراحان عموما نویسنده متن نیستند و وظیفه رعایت حق تکثیر متون را ندارند و در همان حال کار آنها به نوعی وابسته به متن می‌باشد آنها با استفاده از محتویات ساختگی، صفحه گرافیکی خود را صفحه‌آرایی می‌کنند تا مرحله طراحی و صفحه‌بندی را به پایان برند.

% Sometimes questions get separated from their bodies. Use a \newpage to force
% them to wrap to the next page.
\newpage

\question
از محیط  
  \texttt{alphaparts}
  برای استفاده از حروف به جای اعداد استفاده کنید.
  \begin{alphaparts}
    \item
      استفاده از \LaTeX

      لورم ایپسوم یا طرح‌نما (به انگلیسی: Lorem ipsum) به متنی آزمایشی و بی‌معنی در صنعت چاپ، صفحه‌آرایی و طراحی گرافیک گفته می‌شود. طراح گرافیک از این متن به عنوان عنصری از ترکیب بندی برای پر کردن صفحه و ارایه اولیه شکل ظاهری و کلی طرح سفارش گرفته شده استفاده می نماید، تا از نظر گرافیکی نشانگر چگونگی نوع و اندازه فونت و ظاهر متن باشد. معمولا طراحان گرافیک برای صفحه‌آرایی، نخست از متن‌های آزمایشی و بی‌معنی استفاده می‌کنند تا صرفا به مشتری یا صاحب کار خود نشان دهند که صفحه طراحی یا صفحه بندی شده بعد از اینکه متن در آن قرار گیرد چگونه به نظر می‌رسد و قلم‌ها و اندازه‌بندی‌ها چگونه در نظر گرفته شده‌است. از آنجایی که طراحان عموما نویسنده متن نیستند و وظیفه رعایت حق تکثیر متون را ندارند و در همان حال کار آنها به نوعی وابسته به متن می‌باشد آنها با استفاده از محتویات ساختگی، صفحه گرافیکی خود را صفحه‌آرایی می‌کنند تا مرحله طراحی و صفحه‌بندی را به پایان برند.
    \item ???
    \item مفید است!
  \end{alphaparts}

\question
  شماره سوالات بدون توجه به نوع سوال به طور خودکار افزایش می یابند.

  لورم ایپسوم یا طرح‌نما (به انگلیسی: Lorem ipsum) به متنی آزمایشی و بی‌معنی در صنعت چاپ، صفحه‌آرایی و طراحی گرافیک گفته می‌شود. طراح گرافیک از این متن به عنوان عنصری از ترکیب بندی برای پر کردن صفحه و ارایه اولیه شکل ظاهری و کلی طرح سفارش گرفته شده استفاده می نماید، تا از نظر گرافیکی نشانگر چگونگی نوع و اندازه فونت و ظاهر متن باشد. معمولا طراحان گرافیک برای صفحه‌آرایی، نخست از متن‌های آزمایشی و بی‌معنی استفاده می‌کنند تا صرفا به مشتری یا صاحب کار خود نشان دهند که صفحه طراحی یا صفحه بندی شده بعد از اینکه متن در آن قرار گیرد چگونه به نظر می‌رسد و قلم‌ها و اندازه‌بندی‌ها چگونه در نظر گرفته شده‌است. از آنجایی که طراحان عموما نویسنده متن نیستند و وظیفه رعایت حق تکثیر متون را ندارند و در همان حال کار آنها به نوعی وابسته به متن می‌باشد آنها با استفاده از محتویات ساختگی، صفحه گرافیکی خود را صفحه‌آرایی می‌کنند تا مرحله طراحی و صفحه‌بندی را به پایان برند.

% Use \renewcommand{\questiontype}{<text>} to change what word is displayed  before numbered questions
\renewcommand{\questiontype}{تحقیق}
\question
  نوع این سوال متفاوت است!

  لورم ایپسوم یا طرح‌نما (به انگلیسی: Lorem ipsum) به متنی آزمایشی و بی‌معنی در صنعت چاپ، صفحه‌آرایی و طراحی گرافیک گفته می‌شود. طراح گرافیک از این متن به عنوان عنصری از ترکیب بندی برای پر کردن صفحه و ارایه اولیه شکل ظاهری و کلی طرح سفارش گرفته شده استفاده می نماید، تا از نظر گرافیکی نشانگر چگونگی نوع و اندازه فونت و ظاهر متن باشد. معمولا طراحان گرافیک برای صفحه‌آرایی، نخست از متن‌های آزمایشی و بی‌معنی استفاده می‌کنند تا صرفا به مشتری یا صاحب کار خود نشان دهند که صفحه طراحی یا صفحه بندی شده بعد از اینکه متن در آن قرار گیرد چگونه به نظر می‌رسد و قلم‌ها و اندازه‌بندی‌ها چگونه در نظر گرفته شده‌است. از آنجایی که طراحان عموما نویسنده متن نیستند و وظیفه رعایت حق تکثیر متون را ندارند و در همان حال کار آنها به نوعی وابسته به متن می‌باشد آنها با استفاده از محتویات ساختگی، صفحه گرافیکی خود را صفحه‌آرایی می‌کنند تا مرحله طراحی و صفحه‌بندی را به پایان برند.

\renewcommand{\questiontype}{سوال}

\question
  در این محیط ها همچنان میتوانید مواردی همچون لیستهای تودرتو راداشته باشید.
  \begin{alphaparts}
    \item استفاده از \LaTeX
      \begin{enumerate}
        \item  terminal را باز کن
        \item  vim را باز کن
        \item  LaTeX بنویس
      \end{enumerate}
    \item ???
    \item مفید است!
  \end{alphaparts}

  لورم ایپسوم یا طرح‌نما (به انگلیسی: Lorem ipsum) به متنی آزمایشی و بی‌معنی در صنعت چاپ، صفحه‌آرایی و طراحی گرافیک گفته می‌شود. طراح گرافیک از این متن به عنوان عنصری از ترکیب بندی برای پر کردن صفحه و ارایه اولیه شکل ظاهری و کلی طرح سفارش گرفته شده استفاده می نماید، تا از نظر گرافیکی نشانگر چگونگی نوع و اندازه فونت و ظاهر متن باشد. معمولا طراحان گرافیک برای صفحه‌آرایی، نخست از متن‌های آزمایشی و بی‌معنی استفاده می‌کنند تا صرفا به مشتری یا صاحب کار خود نشان دهند که صفحه طراحی یا صفحه بندی شده بعد از اینکه متن در آن قرار گیرد چگونه به نظر می‌رسد و قلم‌ها و اندازه‌بندی‌ها چگونه در نظر گرفته شده‌است. از آنجایی که طراحان عموما نویسنده متن نیستند و وظیفه رعایت حق تکثیر متون را ندارند و در همان حال کار آنها به نوعی وابسته به متن می‌باشد آنها با استفاده از محتویات ساختگی، صفحه گرافیکی خود را صفحه‌آرایی می‌کنند تا مرحله طراحی و صفحه‌بندی را به پایان برند.

% Use the \setcounter{questionCounter}{<x>} to force the question number to a particular question. If your written homework's question start at number 9001,
% use the following
\setcounter{questionCounter}{9000}
\question
  \href{https://www.youtube.com/watch?v=SiMHTK15Pik}{بیشتر از 9000!!!}

  لورم ایپسوم یا طرح‌نما (به انگلیسی: Lorem ipsum) به متنی آزمایشی و بی‌معنی در صنعت چاپ، صفحه‌آرایی و طراحی گرافیک گفته می‌شود. طراح گرافیک از این متن به عنوان عنصری از ترکیب بندی برای پر کردن صفحه و ارایه اولیه شکل ظاهری و کلی طرح سفارش گرفته شده استفاده می نماید، تا از نظر گرافیکی نشانگر چگونگی نوع و اندازه فونت و ظاهر متن باشد. معمولا طراحان گرافیک برای صفحه‌آرایی، نخست از متن‌های آزمایشی و بی‌معنی استفاده می‌کنند تا صرفا به مشتری یا صاحب کار خود نشان دهند که صفحه طراحی یا صفحه بندی شده بعد از اینکه متن در آن قرار گیرد چگونه به نظر می‌رسد و قلم‌ها و اندازه‌بندی‌ها چگونه در نظر گرفته شده‌است. از آنجایی که طراحان عموما نویسنده متن نیستند و وظیفه رعایت حق تکثیر متون را ندارند و در همان حال کار آنها به نوعی وابسته به متن می‌باشد آنها با استفاده از محتویات ساختگی، صفحه گرافیکی خود را صفحه‌آرایی می‌کنند تا مرحله طراحی و صفحه‌بندی را به پایان برند.


\section{}
همچنین میتوانید از ماکروی 
\texttt{section}
برای شروع یک سوال استفاده کنید.

  لورم ایپسوم یا طرح‌نما (به انگلیسی: Lorem ipsum) به متنی آزمایشی و بی‌معنی در صنعت چاپ، صفحه‌آرایی و طراحی گرافیک گفته می‌شود. طراح گرافیک از این متن به عنوان عنصری از ترکیب بندی برای پر کردن صفحه و ارایه اولیه شکل ظاهری و کلی طرح سفارش گرفته شده استفاده می نماید، تا از نظر گرافیکی نشانگر چگونگی نوع و اندازه فونت و ظاهر متن باشد. معمولا طراحان گرافیک برای صفحه‌آرایی، نخست از متن‌های آزمایشی و بی‌معنی استفاده می‌کنند تا صرفا به مشتری یا صاحب کار خود نشان دهند که صفحه طراحی یا صفحه بندی شده بعد از اینکه متن در آن قرار گیرد چگونه به نظر می‌رسد و قلم‌ها و اندازه‌بندی‌ها چگونه در نظر گرفته شده‌است. از آنجایی که طراحان عموما نویسنده متن نیستند و وظیفه رعایت حق تکثیر متون را ندارند و در همان حال کار آنها به نوعی وابسته به متن می‌باشد آنها با استفاده از محتویات ساختگی، صفحه گرافیکی خود را صفحه‌آرایی می‌کنند تا مرحله طراحی و صفحه‌بندی را به پایان برند.

\section*{Using the \texttt{section} Macro}
\texttt{section*}
هم کار میکند.

  لورم ایپسوم یا طرح‌نما (به انگلیسی: Lorem ipsum) به متنی آزمایشی و بی‌معنی در صنعت چاپ، صفحه‌آرایی و طراحی گرافیک گفته می‌شود. طراح گرافیک از این متن به عنوان عنصری از ترکیب بندی برای پر کردن صفحه و ارایه اولیه شکل ظاهری و کلی طرح سفارش گرفته شده استفاده می نماید، تا از نظر گرافیکی نشانگر چگونگی نوع و اندازه فونت و ظاهر متن باشد. معمولا طراحان گرافیک برای صفحه‌آرایی، نخست از متن‌های آزمایشی و بی‌معنی استفاده می‌کنند تا صرفا به مشتری یا صاحب کار خود نشان دهند که صفحه طراحی یا صفحه بندی شده بعد از اینکه متن در آن قرار گیرد چگونه به نظر می‌رسد و قلم‌ها و اندازه‌بندی‌ها چگونه در نظر گرفته شده‌است. از آنجایی که طراحان عموما نویسنده متن نیستند و وظیفه رعایت حق تکثیر متون را ندارند و در همان حال کار آنها به نوعی وابسته به متن می‌باشد آنها با استفاده از محتویات ساختگی، صفحه گرافیکی خود را صفحه‌آرایی می‌کنند تا مرحله طراحی و صفحه‌بندی را به پایان برند.

\end{document}
